% This is an example of how to format a thesis with LaTeX the
% simplest possible way (permitted beginning in 2010).
% NO UGA STYLE SHEET IS NEEDED.

\documentclass[12pt]{report}
\usepackage{fullpage}
\usepackage{setspace}\doublespacing    % important!
\textfloatsep 0.75in                   % important with double spacing

\begin{document}

% Make the official abstract page
\newpage
\thispagestyle{empty}
\vspace*{18pt}
\begin{center}
\textsc{This is the Title of the Thesis\\Broken Across Two Lines}\\[18pt]
by\\[18pt]
\textsc{Author's Name Here}\\[12pt]
(Under the direction of Professor's Name Here)\\[12pt]
\textsc{Abstract}
\end{center}
This is the abstract, a brief summary of the contents of the thesis.  It 
is limited to 150 words in length.

The abstract page(s) are not numbered and are not necessarily included in a
printed and bound copy of the thesis.  The index words at the bottom of the
abstract should be chosen carefully, preferably with the help of your colleagues.

% Display the index words (this is a bit fancy):
\begin{list}{\sc Index words:\hfill}{\labelwidth 1.2in\leftmargin 1.4in\labelsep 0.2in}
\item 
\begin{flushleft}\singlespacing
Index word or phrase,
Index word or phrase,
Index word or phrase,
Index word, 
Index word,
Index word
\end{flushleft}
\end{list}



% Make the official title page
\newpage
\thispagestyle{empty}
\vspace*{18pt}
\begin{center}
\textsc{This is the Title of the Thesis\\Broken Across Two Lines}\\[18pt]
by\\[18pt]
\textsc{Author's Name Here}\\[12pt]
B.A., University of Somewhere, 1998\\
M.A., University of Somewhere Else, 2001\\
\vfill
A Dissertation Submitted to the Graduate Faculty \\
of The University of Georgia in Partial Fulfillment \\
of the \\
Requirements for the Degree \\[10pt]
\textsc{Doctor of Philosophy}\\[36pt]
\textsc{Athens, Georgia}\\[18pt]
2010
\end{center}

% Make the copyright page
\newpage
\thispagestyle{empty}
\vspace*{5.5in}
\begin{center}
\copyright 2010 \\
Author's Name Here
\end{center}

% Make the approval page
\newpage
\thispagestyle{empty}
\vspace*{18pt}
\begin{center}
\textsc{This is the Title of the Thesis\\Broken Across Two Lines}\\[18pt]
by\\[18pt]
\textsc{Author's Name Here}
\end{center}
\vfill
\begin{flushleft}\singlespacing
\hskip 200pt {Approved:}\\
\vskip 12pt
% Two major professors.  If you have only one, change word to "Professor".
\hspace*{200pt}\makebox[100pt][l]{Major Professors:}Professor's Name\\
\hspace*{200pt}\makebox[100pt][l]{~                }Professor's Name\\
\vskip 12pt
% Committee (use as many lines as needed)
\hspace*{200pt}\makebox[100pt][l]{Committee:       }Member's Name\\
\hspace*{200pt}\makebox[100pt][l]{~                }Member's Name\\
\hspace*{200pt}\makebox[100pt][l]{~                }Member's Name\\
% Approval words
\vfill
Electronic Version Approved:\\[12pt]
Dean's Name Here\\
Dean of the Graduate School\\
The University of Georgia\\
June 2010
\end{flushleft}


% Now we begin the regular LaTeX document.
% You may want to have a regular LaTeX title page here...
\title{\bf This is the Title of the Thesis\\Broken Across Two Lines}
\author{Author's Name Here}
\maketitle 

\chapter*{Acknowledgments}
You can have acknowledgments or a preface here.

\tableofcontents
\listoffigures  % if any
\listoftables % if any

\chapter{This is the First Chapter}

\section{This is a section in the first chapter}

This is an example of how to format a thesis with LaTeX in the simplest possible way.
This is a simple example.  This is a simple example and will be very boring to read.
This is an example of how to format a thesis with LaTeX in the simplest possible way.

This is a simple example.  This is a simple example and will be very boring to read.
This is an example of how to format a thesis with LaTeX in the simplest possible way.
This is a simple example.  This is a simple example and will be very boring to read.
This is an example of how to format a thesis with LaTeX in the simplest possible way.

\begin{figure}
\centerline{[ You could put a picture here. ]}
\caption{Example of a figure.}
\end{figure}

\begin{table}
\caption{Example of a table.}
\centerline{[The contents of the table would go here.]}
\end{table}

This is a simple example.  This is a simple example and will be very boring to read.
This is an example of how to format a thesis with LaTeX in the simplest possible way.
This is a simple example.  This is a simple example and will be very boring to read.
This is an example of how to format a thesis with LaTeX in the simplest possible way.

This is a simple example.  This is a simple example and will be very boring to read.
This is an example of how to format a thesis with LaTeX in the simplest possible way.
This is a simple example.  This is a simple example and will be very boring to read.

Proceed with more sections and chapters, bibliography, etc., after this.


\chapter{What \LaTeX\ is all about}

This document is an example of how to format a thesis or dissertation
using \LaTeX\  and get results acceptable at The University of Georgia.

\LaTeX\ (with its parent \TeX)
has two major advantages for academic use.  First, to a remarkable
degree it makes design decisions automatically.  The author supplies
only the words of a text, and \LaTeX\ places them on the page in an
aesthetic manner, avoiding rivers and awkward breaks.  In this respect
LaTeX is like a very intelligent typist or typesetter.

Second, \LaTeX\ can typeset complex mathematical formulas such as
\[
\sum_{i=1}^{\infty} x^{y+z} = \frac{p+q+r}{s+t+u+v}
\]
both displayed (as shown above) and in the text, as in
$\sum_{i=1}^{\infty} x^{y+z} = \frac{p+q+r}{s+t+u+v}$.
This makes \TeX\ and \LaTeX\ indispensable for mathematicians, physicists,
and the like.

LaTeX also has built--in formats for other kinds of displayed material
such as verse,
\begin{verse}
Freude, sch\"{o}ne G\"{o}tterfunken  \\
Tochter aus Elysium,                 \\
Wir betreten, feuertrunken,          \\
Himmlische, dein Heiligtum!          \\
Deine Zauber binden wieder            \\
Was der Mode streng geteilt\dots
\end{verse}
and direct quotations:
\begin{quote}
The society that scorns excellence in plumbing, because plumbing is a
humble activity, and tolerates shoddiness in philosophy, because
philosophy is an exalted activity, will have neither good plumbing nor
good philosophy.  Neither its pipes nor its theories will hold water.\\
\hspace*{\fill} --- John Gardner, {\em Excellence}
\end{quote}
If you wish, quotes and other displayed material can be single--spaced;
here is an example of how that is achieved:
\begin{verse}
\begin{singlespace}
Yes, I wrote ``The Purple Cow.'' \\
I'm sorry now I wrote it. \\
But I can tell you anyhow \\
I'll kill you if you quote it! \\
\hfill --- Anonymous?
\end{singlespace}
\end{verse}
Well, maybe it's not as anonymous as it looks.

Computer scientists use LaTeX's ``verbatim'' format to display portions
of computer programs in text, like this:
\begin{singlespace}\begin{verbatim}
1   GO TO 2
2   GO TO 1
3   PRINT "THIS STATEMENT WILL NEVER EXECUTE"
4   END
\end{verbatim}\end{singlespace}
There you have it.%
\footnote{This is a footnote. Notice that footnotes are single--spaced
even though the text is double--spaced. Long footnotes are discouraged;
either make important points in the text or leave them out.}

Whenever you quote parts of a computer program in English text, they
should be set off by using typewriter type.  The \verb"PRINT" statement
in the program above will never execute because the \verb"GO TO"
statement above it keeps execution from reaching it.




\end{document}


