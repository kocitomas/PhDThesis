% This is an example of how to format a thesis with LaTeX the
% simplest possible way (permitted beginning in 2010).
% NO UGA STYLE SHEET IS NEEDED.

\documentclass[12pt]{report}
\usepackage{fullpage}
\usepackage{setspace}\doublespacing    % important!
\textfloatsep 0.75in                   % important with double spacing

\begin{document}

% Make the official abstract page
\newpage
\thispagestyle{empty}
\vspace*{18pt}
\begin{center}
\textsc{Structure Forming Processes in\\Mesoscopic Polymer Systems}\\[18pt]
by\\[18pt]
\textsc{Tomas Koci}\\[12pt]
(Under the direction of Michael Bachmann)\\[12pt]
\textsc{Abstract}
\end{center}
This is going to be the best abstract ever :)

% Display the index words (this is a bit fancy):
\begin{list}{\sc Index words:\hfill}{\labelwidth 1.2in\leftmargin 1.4in\labelsep 0.2in}
\item 
\begin{flushleft}\singlespacing
Index word or phrase,
Index word or phrase,
Index word or phrase,
Index word, 
Index word,
Index word
\end{flushleft}
\end{list}



% Make the official title page
\newpage
\thispagestyle{empty}
\vspace*{18pt}
\begin{center}
\textsc{Structure Forming Processes in\\Mesoscopic Polymer Systems}\\[18pt]
by\\[18pt]
\textsc{Tomas Koci}\\[12pt]
B.A., The Juilliard School, 2008\\
\vfill
A Dissertation Submitted to the Graduate Faculty \\
of The University of Georgia in Partial Fulfillment \\
of the \\
Requirements for the Degree \\[10pt]
\textsc{Doctor of Philosophy}\\[36pt]
\textsc{Athens, Georgia}\\[18pt]
2016
\end{center}

% Make the copyright page
\newpage
\thispagestyle{empty}
\vspace*{5.5in}
\begin{center}
\copyright 2016 \\
Tomas Koci
\end{center}

% Make the approval page
\newpage
\thispagestyle{empty}
\vspace*{18pt}
\begin{center}
\textsc{Structure Forming Processes in\\Mesoscopic Polymer Systems}\\[18pt]
by\\[18pt]
\textsc{Tomas Koci}
\end{center}
\vfill
\begin{flushleft}\singlespacing
\hskip 200pt {Approved:}\\
\vskip 12pt
% Two major professors.  If you have only one, change word to "Professor".
\hspace*{200pt}\makebox[100pt][l]{Major Professor:}Michael Bachmann\\
\vskip 12pt
% Committee (use as many lines as needed)
\hspace*{200pt}\makebox[100pt][l]{Committee:       }Steven P. Lewis\\
\hspace*{200pt}\makebox[100pt][l]{~                }Heinz-Bernd Schuttler\\
% Approval words
\vfill
Electronic Version Approved:\\[12pt]
Alan Dorsey\\
Dean of the Graduate School\\
The University of Georgia\\
July 2016
\end{flushleft}


% Now we begin the regular LaTeX document.
% You may want to have a regular LaTeX title page here...
\title{\bf Structure Forming Processes in\\Mesoscopic Polymer Systems}
\author{Tomas Koci}
\maketitle 

\chapter*{Acknowledgments}
Mention Michael Bachmann, Steven Lewis, Heinz Schuttler, D.P. Landau, Jeff Mike and Shan-Ho, finally all the Links and my family

\tableofcontents
\listoffigures  % if any
\listoftables % if any

\chapter{Introduction}
Kickass Intro...

\chapter{Elements of Statistical Mechanics}
Statistical mechanics explains the microscopic origins of macroscopic properties of systems with large numbers of degrees of freedom. The exact solution for a phase space trajectory of a complex system requires enormous computational efforts and contains little useful information. On the other hand, collective properties such as entropy, pressure, or temperature often display relatively simple behavior. The formalism of statistical mechanics allows us to study these properties by considering the average behavior of a large number of identically prepared systems; the statistical ensemble. It is well established that in the thermodynamic limit all ensembles are equivalent. However this is emphatically not true for intrinsically finite systems for which the choice of an ensemble is non-trivial. Therefore, I shall first discuss several prominent statistical ensembles starting with the most fundamental one; the \textit{microcanonical ensemble}.

\section{The microcanonical ensemble}
As a starting point, let us consider a mechanically and adiabatically isolated system with a constant number of particles $(N)$, volume $(V)$, and energy $(E)$. At any given moment, the system is to be found in one of the accessible microstates $\mu$ which are represented by points in the $6N$ dimensional phase-space. At a fixed energy $E$, the allowed microstates are constrained to the surface of constant energy $H(\mu) = E$, where $H(\mu)$ is the Hamiltonian of the system. The total number of microstates corresponding to a macrostate with a fixed energy $E$ is given by the density of states
\begin{equation}
g(E) = \int dPdQ \delta(E - H(P,Q)).
\end{equation} 
Assuming that there are no other conserved quantities, so that that the system is ergodic, all the microstates have equal a priori probabilities. Therefore the microcanonical probability distribution can be written as 
\begin{equation}
p(\mu)_{E} = 1/g(E) or 0
\end{equation}
The expectation value of an observable in the microcanonical ensemble is then found by evaluating
\begin{equation}
<O>_{E} = 1/g(E)\int dPdQ  O(P,Q)\delta(E - H(P,Q))
\end{equation}


 This is one of the fundamental postulates of statistical mechanics.
\section{The canonical ensemble}
\section{Generalized ensembles}

\chapter{Computational Methods}
\section{Markov chain Monte Carlo}
\subsection{Master equation and detailed balance}
\subsection{Metropolis sampling}
\section{Generalized ensemble Monte Carlo}
\subsection{Parallel tempering}
\subsection{Multiple Gaussian modified ensemble}
\subsection{Histogram reweighting methods}
\subsection{Multicanonical sampling}
\section{Simple Monte Carlo updates}

\chapter{Coarse-grained Homopolymer Model}
\section{Flexible elastic homopolymer}
\section{Interacting homopolymers}

\chapter{Confinement Effects on Structural Transitions in Flexible Homopolymers}
\section{Introduction}
\section{Canonical analysis}
\section{Inflection-point analysis}
\section{Hyper-phase diagrams}

\chapter{Impact of Bonded Interactions on the Ground-State Geometries of Flexible Homopolymers}
\section{Structural order parameters}
\section{15-mer}
\section{55-mer}

\chapter{Aggregation of Flexible Elastic Homopolymers}
\section{Introduction}
\section{Microcanonical analysis}
\subsection{Subphases and subphase transitions}
\subsection{Missing subphases and translational entropy}
\subsection{Density effects on the latent heat}

\chapter{Summary and Outlook}










\begin{figure}
\centerline{[ You could put a picture here. ]}
\caption{Example of a figure.}
\end{figure}

\begin{table}
\caption{Example of a table.}
\centerline{[The contents of the table would go here.]}
\end{table}









\end{document}


